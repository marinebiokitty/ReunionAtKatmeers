\documentclass[blue]{Katmeers}
\begin{document}
\name{\bRules{}}

\updatemacro{\cHarry}{
  \nickname{Drew Carpenter}
  \mapnickinformal
}

\section{Combat System:}

A Reunion at Katmeers will be run on the Priority System, which was devised by Mark Berghausen and Victor Cepeda.

\subsection*{Combat Abilities}
Each player is assigned a number of combat abilities they have access to. Combat abilities have the following parts:
Priority \#: (abbreviated P\#) This is a measure of how quickly your attack happens; attacks with lower priority \#'s happen first. (Attacks cannot have a priority lower than 1.)\\
\indent Name: The name of the ability being used.\\
\indent Description: What the attack looks like, included for role-playing flavor.\\
\indent Effect: The system effect of this ability (how much damage it deals, etc). Text will be in bold\\
\indent Example: P3 Jab: Fling a quick punch at your opponent. {\bf Deal 1 Damage to a single target.}

\subsection*{The Combat Round}
First, all players involved in combat select the ability they will use that round, as well as the target(s) for their action. The attacks resolve in order of smallest priority to largest. If an attack hits you before your attack resolves, your attack is interrupted (doesn't happen this round). Attacks with the same priority happen simultaneously. Combat continues in this fashion until all players on one side are wounded.

\subsection*{Improvised Weapons}
In addition to their other abilities, all characters have the ability to wield found objects (chairs, bottles, grimoires, etc.) as weapons, using the following attack:\\
P7 Brawl: A nonmagical weapon attack.\\
You pick up the closest object and give your opponent a hearty smack with it. Does 2 damage.  May have other effects as well.  Please pick up the object OOC so it is clear what is being used; please to do not actually wield it weapons-wise.

\subsection*{Fleeing}
All characters have the following:\\
\indent P5 Flee: Run away from combat. You are removed from combat.\\
You can only successfully flee combat if no one hits you with an attack of priority 4 or lower.

\subsection*{Wounds}
Each character is assigned a \# of HP between 7 and 13. If your HP are ever reduced to 0, you are Wounded. Being Wounded means you fall to the floor, incapacitated and are removed from combat. You remain in this state for a minute after the combat that Wounded you ends. Each time you are Wounded, reduce your maximum HP by 2.

After one minute, you wake up and regain HP equal to half your new total, rounded down.

\subsection*{Death}
Any player may attempt to kill a Wounded player. To do this.
\begin{enumerate}
	\item There must be no combat.
	\item You must count to ``10'' slowly, using the phrase ``Killing Blow \#''. ie: ``Killing Blow 1'', ``Killing Blow 2''\ldots{} ``Killing blow 10.'' At any time during this count, another player may stop you by saying ``I stop you.''
	\item If no one stops you before you reach ten, then your victim is dead and becomes a ghost. (They should report to the GM.) Their body will vanish after five minutes.  It's a wizard thing.
\end{enumerate}

\section{Other}

\subsection*{Ghosts}
By default, ghosts are invisible and incorporeal, and can only interact with other ghosts.  Characters who are ghosts will carry a lantern with them for quick visual reference. (Or some sort of facsimile of a lantern.) It is easy to tell ghosts apart, but not necessarily obvious who they were when they were alive.

\subsection*{Cantrips}
All of you wizard and fairy characters, you can cast minor cosmetic magical effects (phantom music, small and obvious illusions, other flavor-rich and not game-affecting things) at will.

Similarly, if you want to have random useless magical (or mundane!) items that I have not included on your list, feel free to make it up and reference it.

\subsection*{Healing}
All Katmeers students are trained in the healing arts.  Unless otherwise noted, you may heal any player (including yourself and Wounded players) up to their new maximum HP after five minutes of spellcasting.  Both the caster and the damaged player may engage in dialogue while this is going on, but neither party may move about or take any other actions.

Only wizards are capable of working healing magic, unless otherwise noted on your character sheet.  Everyone else will need to seek out a wizard to do their healing for them.

If you finish a combat having taken damage but not being Wounded (e.g. 5/10 HP remaining), you will continue to have that damage until you get healing. If you have been Wounded so many times that your maximum HP is 0, then you die and become a ghost.

\section{Miscellaneous}
There are some abilities that affect priority; no ability's priority can be made lower than P1.

Some abilities are labeled ``Dark.'' Dark Magic is extremely obvious; you will not be able to conceal from any observers that you just cast a Dark spell, and in fact must announce OOC that you have used Dark Magic.


\end{document}
